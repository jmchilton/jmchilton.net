\documentclass{beamer}
\usepackage{beamerthemesplit}
\setbeamercovered{dynamic}


\AtBeginSection[]
{
   \begin{frame}
       \frametitle{Outline}
       \tableofcontents[currentsection]
   \end{frame}
}

\usetheme{Warsaw}
\usefonttheme[onlysmall]{structurebold}

% \usepackage[usenames]{color}

\title{Template Metaprogramming for Haskell}

\author{John Chilton}
\date{March 5, 2007}
\begin{document}

% \frame{\titlepage}
\frame{
\begin{center}
\LARGE{Template Metaprogramming for Haskell} \\ 
~\\
\Large{Tim Sheard and Simon Peyton Jones} \\
~\\
~\\
\Large{Presented by John Chilton} \\
\Large{March 5, 2007}
\scriptsize{\url{http://www.jmchilton.net/index.html?page=geek:template_haskell}}
\end{center}
}

\frame{\tableofcontents}
\section{Introduction}

\subsection{Introduction to Haskell, Metaprogramming, and \ttfamily{printf}\sffamily{}}
\frame{
\frametitle{Haskell}
\begin{center}\Large
See \ttfamily{haskell.hs}\sffamily{ and } \ttfamily{haskell-monad.hs}
\end{center}
}

\frame{
\frametitle{Metaslide}
The paper is out of date, a lot of code doesn't work and many names and concepts have changed slightly. I will present an up to date version of Template Haskell, with some modified and new examples that actually compile.
~\\
\scriptsize{\url{http://www.jmchilton.net/index.html?page=geek:template_haskell}}
}





\frame{
\frametitle{Metaprogramming}
\begin{block}{What?}
Metaprogramming facilities allows the programmer to analyze and compute parts of programs.
\end{block}
\begin{block}{Why?}
\begin{itemize}
\item Compile-time, programmable optimizations and analysis
\item Mechanism for creating new abstractions
\item Application: Embedded domain specific languages
\end{itemize}
\end{block}
}

\frame[<+->]{
\frametitle{Introducing the \ttfamily{printf}\sffamily{ Example}}
\begin{alertblock}{The Goal}
\ttfamily{printf "Error: \%s on line \%d." msg line}
\end{alertblock}
\begin{block}{Problems}<2->
\begin{itemize}
 \item<2-> Cannot be implemented in ordinary Haskell
 \item<2-> Interdependence of types
 \item<2-> Variable number of arguments
\end{itemize}
\end{block}
\begin{block}{Template Haskell Solution}<3->
Generate appropriate procedure at compile time.
\end{block}
}

\subsection{The \ttfamily{\$(...)}\sffamily{ Operator}}
\frame{
\frametitle{\ttfamily{\$(...)}\sffamily{ Syntax}}
\begin{center}
\ttfamily{\$(...)} \\
\LARGE{$\Downarrow$}\normalsize \\
\sffamily{ Must contain valid Haskell expression} \\
\LARGE{$\Downarrow$}\normalsize \\
Expression must yield Haskell syntax tree data \\
\LARGE{$\Downarrow$}\normalsize \\
Evaluated at runtime and inserted into program
\end{center}
}

\frame{
\frametitle{Two Types of Spliceable Data}
\begin{description}
 \item[ExpQ] Syntax trees corresponding to valid Haskell expressions (literals, variables, function application, conditionals, etc.). Called Expr in paper.
 \item[DecQ] Syntax trees corresponding to a valid Haskell declarations (functions, values, data types, classes, etc.). Called Decl in paper.
%  \item[TypeQ] AST corresponding to a valid Haskell type. 
\end{description}
}

% % % % % % % % % % % % % % % % % % % % % % % % % % % % % % 
% % % % % % % % % % % % % % % % % % % % % % % % % % % % % % 
% % % % % % % % % % % % % % % % % % % % % % % % % % % % % % 
% % % % % % % % % % % % % % % % % % % % % % % % % % % % % % 
% % % % % % % % % % % % % % % % % % % % % % % % % % % % % % 

\section{Building Haskell Syntax Trees}
\subsection{Example I: \ttfamily{sel}\sffamily{}}
\frame{
\frametitle{The Problem}
\begin{Example}
\ttfamily{> listSelect 2 [1,2,3,4,5]}  \\
\ttfamily{3} \\
\ttfamily{> :type listSelect} \\
\ttfamily{Int -> [a] -> a}  \\
\ttfamily{> sel 2 (1,2,3,4,5) --cannot be implemented} \\ 
\end{Example}

\begin{alertblock}{The Problem}<2->
\sffamily{No generic tuple, (a), type corresponding to list type, [a]. }
\end{alertblock}
}

\frame{
\frametitle{sel}
\begin{Example} \Large 
\ttfamily{> \$(sel 3 5) (1,2,3,4,5)} \\
\ttfamily{3} \\
\ttfamily{> ($\backslash$(a1,a2,a3,a4,a5) -> a3) (1,2,3,4,5)} \\
\ttfamily{3}
\end{Example}
% \begin{block}{How sel will work}
% \ttfamily{\$(sel m n)} \\
% \sffamily{At compile time will generate} \\
% \ttfamily{($\backslash$(a1,a2,a3,\textbf{...},an) -> am)}
% \end{block}
}

\frame{
\frametitle{Implementing \ttfamily{sel}\sffamily{}}
\begin{Example}
\begin{center}\ttfamily
(\textbackslash (a1, a2, a3, a4) -> a2)
\end{center}
\end{Example}
\begin{itemize}
\item \ttfamily{lamE :: [PatQ] -> ExpQ -> ExpQ}
\item \ttfamily{tupP :: [PatQ] -> PatQ}
\item \ttfamily{varP :: Name -> PatQ}
\item \ttfamily{varE :: Name -> ExpQ}
\item \ttfamily{mkName :: String -> Name}
\end{itemize}
\begin{block}{Implementation}\sffamily
See \ttfamily Sel.hs
\end{block}
}

\subsection{Example II: \ttfamily{printf}\sffamily{}}
\frame[<+->]{
\frametitle{printf}
\begin{block}{Type}
Implement printf with type - \ttfamily{String -> ExpQ}
\end{block}

\begin{Example}
\ttfamily{\$(printf "Error: \%s on line \%d.") msg line} \\
\end{Example}

\begin{block}{Compile Time Expansion}
\ttfamily{(\textbackslash s0 -> \textbackslash n1 -> \\
~~~"Error: " ++ s ++ " on line " ++ show n) \\
~msg line}
\end{block}
}



\frame{
\begin{center}\ttfamily{appE :: ExpQ -> ExpQ -> ExpQ}\end{center}
\begin{block}{Example}<2->
\ttfamily{showE :: ExpQ -> ExpQ} \\
\ttfamily{showE e = appE (varE (mkName "show")) e}
\end{block}
}

\frame{
\begin{center}\ttfamily{infixE::(Maybe ExpQ) -> ExpQ -> (Maybe ExpQ) -> ExpQ}\end{center}
\begin{block}{Haskell's Maybe Type}<2->
\ttfamily{data Maybe a = Nothing | Just a}
\end{block}
\begin{Example}<2->
\ttfamily{> map (+ 2) [1..5]} \\
\ttfamily{[3,4,5,6,7]} \\
\end{Example}
\begin{block}{concatE}<3->
\ttfamily{concatE :: ExpQ -> ExpQ -> ExpQ} \\
\ttfamily{concatE e1 e2 =}\\
\ttfamily{~~~~infixE (Just e1) (varE (mkName "++")) (Just e2)}
\end{block}
}

\frame{
\begin{block}{\ttfamily{printf}\sffamily{ Implementation - See }\ttfamily{Printf1.hs}}
\ttfamily
data Format = D | S | L String \\
~\\
printf :: String -> ExpQ \\
printf s = gen (parse s) (litE (StringL "")) \\
~\\
parse :: String -> [Format] \\
... \\
gen :: [Format] -> ExpQ -> ExpQ \\
...
\end{block}
\begin{Example}
\ttfamily
> parse "Error: \%s on line \%d." \\
~[L "Error: ", S, L " on line ", D, L "."]
\end{Example}
}


\frame{
\begin{block}{main1.hs}\ttfamily
module Main where \\
import Printf1 \\
~\\
errorAt :: String -> Int -> String  \\
errorAt msg line =\\
~~~~\$(printf "Error: \%s on line \%d") msg line \\
~\\
main = putStrLn (errorAt "Undeclared variable" 314)
\end{block}
\begin{Example}\ttfamily
@jl \% ghc --make -fth main1.hs \\
@jl \% main1 \\
Error: Undeclared variable on line 314 \\
\end{Example}
}

\frame{
\begin{block}{main2.hs}\ttfamily
module Main where \\
import Printf1 \\
~\\
errorVar :: String -> String -> String \\
errorVar msg var = \\
~~~~ \$(printf "Error \%s with variable \%s") msg var \\
~\\
main = putStrLn (errorVar "Undeclared variable" "fo")
\end{block}
\begin{Example}<2->\ttfamily
@jl \% ghc --make -fth main2.hs \\
@jl \% main2 \\
Error fo with variable fo \\
\end{Example}
}

\frame{
\begin{block}{main2.hs}\ttfamily
module Main where \\
import Printf1 \\
~\\
errorVar :: String -> String -> String \\
errorVar msg var = \\
~~~~ \$(printf "Error \%s with variable \%s") msg var \\
~\\
main = putStrLn (errorVar "Undeclared variable" "fo")
\end{block}
\begin{alertblock}{Problem - Generated Code}\ttfamily
(\textbackslash s -> \textbackslash s -> \\
~~~"" ++ "Error: " ++ s ++ " with variable " ++ s) \\
~msg var
\end{alertblock}
}

\frame{
\frametitle{gensym operator}\sffamily
\begin{itemize}
\item Common in Lisp \& Scheme dialects
\item Generates a fresh variable name, not clashing with any existing variables
\item Can be used to simulate lexical scoping
\item Template Haskell now calls it \ttfamily{qNewName}
\end{itemize}
\begin{center}\ttfamily
qNewName :: String -> Q Name
\end{center}
}

\frame{
\frametitle{Using gensym}
\begin{alertblock}{Broken - \ttfamily{Printf1.hs}}\ttfamily
gen (D : xs) x = \\
~~let body = concatE x (showE (varE (mkName "n"))) \\
~~in lamE [varP (mkName "n")] (gen xs body)
\end{alertblock}

\begin{block}{Fixed - \ttfamily{Printf2.hs}}\ttfamily
gen (D : xs) x = \\
~~do \{ n <- qNewName "n"; \\ 
~~~~~~~let body = concatE x (showE (varE n)) \\ 
~~~~~~~in lamE [varP n] (gen xs body) \} 
\end{block}
}

\frame{
\begin{block}{main3.hs}\ttfamily
module Main where\\
import Printf2\\
~\\
errorVar :: String -> String -> String \\
errorVar msg var = \\
~~~\$(printf "Error \%s with variable \%s") msg var \\
~\\
main = putStrLn (errorVar "Undeclared variable" "fo")\\
\end{block}

\begin{Example}\ttfamily
@jl \% ghc --make -fth main3.hs\\
@jl \% main3\\
Error Undeclared variable with variable fo
\end{Example}
}

% % % % % % % % % % % % % % % % % % % % % % % % % % % % % % 
% % % % % % % % % % % % % % % % % % % % % % % % % % % % % % 
% % % % % % % % % % % % % % % % % % % % % % % % % % % % % % 
% % % % % % % % % % % % % % % % % % % % % % % % % % % % % % 
% % % % % % % % % % % % % % % % % % % % % % % % % % % % % % 

\section{Some Additonal Features}
\subsection{The Quasi-Quote}
\frame{
\begin{itemize} \ttfamily\Large
\item ~\$(~) $\widetilde{::}$ ExpQ -> "Haskell code"
\item ~[\textbar ~ \textbar] $\widetilde{::}$ "Haskell Code" -> ExpQ
\end{itemize}
}


\frame{
\frametitle{Toy Example}
\begin{Example}\ttfamily
[| let x = 7\\
~~~in show x |]
\end{Example}

\begin{block}{Result}\ttfamily
Q (LetE [ValD (VarP x\_0) \\
~~~~~~~~~~~~~~(NormalB (LitE (IntegerL 7)))\\ 
~~~~~~~~~~~~~~[]\\
~~~~~~~~] \\
~~~~~~~~(AppE (VarE GHC.Show.show) (VarE x\_0)) \\
~~~)
\end{block}
}

\frame{
\frametitle{More Examples}
\begin{block}{\ttfamily{showE}}\ttfamily
\ttfamily{showE :: ExpQ -> ExpQ} \\
% \ttfamily{--showE e = appE (varE (mkName "show")) e}\\
\ttfamily{showE e = [| show \$(e) |]}
\end{block}
\begin{block}{\ttfamily{concatE}}\ttfamily
\ttfamily{concatE :: ExpQ -> ExpQ -> ExpQ} \\
\ttfamily{concatE e1 e2 = }\\
\ttfamily{~~~~[| \$(e1) ++ \$(e2) |]}
\end{block}
}

\frame{
\frametitle{\ttfamily{printf}\sffamily{ with Quasi-Quotes}}
\begin{block}{\ttfamily gen}\ttfamily
gen [] x = x \\
gen (D : xs) x = \\
~~~~[| \textbackslash n -> \$(gen xs [| \$(x)++show n |]) |] \\
gen (S : xs) x = \\
~~~~[| \textbackslash s -> \$(gen xs [| \$(x)++s |]) |] \\
gen (L s : xs) x = \\
~~~~gen xs [| \$(x) ++ \$(lift s) |] 
\end{block}
\begin{alertblock}{Scoping}<2->
Variables in the Quasi-Quote are statically scoped.
\end{alertblock}
}

\frame{
\frametitle{Other Quasi-Quotes}
\begin{itemize} \ttfamily\Large
\item ~\$(~) $\widetilde{::}$ ExpQ -> "Haskell code"
\item ~[\textbar ~ \textbar] $\widetilde{::}$ "Haskell Code" -> ExpQ
\item ~[d\textbar ~ \textbar] $\widetilde{::}$ "Haskell Code" -> DecQ
\item ~[t\textbar ~ \textbar] $\widetilde{::}$ "Haskell Code" -> TypeQ
\item ~[p\textbar ~ \textbar] is not implemented, breaks a lot of code from the paper
\end{itemize}
}

\subsection{Splicing Declarations}
\frame{
\frametitle{\sffamily{Splicing Declarations - }\ttfamily{gen\_sels}}
\begin{block}{\ttfamily mainsel2.hs}\ttfamily
module Main where \\
import Sel \\
~\\
\$(gen\_sels 7) \\
-- Defines procedures sel1of7, sel2of7, ..., sel7of7 \\
~\\
main = putStrLn (sel4of7 (1,"b",3,"d",sqrt,"f",1))
\end{block}
\begin{Example}\ttfamily\small
@jl \% ghc -fth --make mainsel2.hs \\
@jl \% mainsel2 \\
d
\end{Example}
}

\subsection{Reification}
\frame{
\frametitle{reify}
\begin{center}
\ttfamily{reify :: Name -> InfoQ}
\end{center}
\begin{block}{\ttfamily reify}\sffamily
Lookups up the compiler's information about Name. The name can be the name of a variable, data type, class, etc.
\end{block}
\begin{alertblock}{Note}
Everything about reification in the paper is out of date. \ttfamily{reify}\sffamily{ encapsulates many of the procedures mentioned, and other functionality such as }\ttfamily{reifyOpt}\sffamily{ or }\ttfamily{reifyLocn}\sffamily{ are not implemented.}
\end{alertblock}
}

% TyConI (DataD [] RLP.Point [] [NormalC RLP.Pt [(NotStrict,ConT GHC.Real.Rational),(NotStrict,ConT GHC.Real.Rational)]] [])
% DataConI RLP.Pt (AppT (AppT ArrowT (ConT GHC.Real.Rational)) (AppT (AppT ArrowT (ConT GHC.Real.Rational)) (ConT RLP.Point))) RLP.Point (Fixity 9 InfixL)
% VarI RLP.x (ConT GHC.Float.Double) Nothing (Fixity 9 InfixL)
% VarI RLP.foo (ForallT [a_1627397870] [AppT (ConT GHC.Num.Num) (VarT a_1627397870)] (AppT (AppT ArrowT (VarT a_1627397870)) (VarT a_1627397870))) Nothing (Fixity 9 InfixL)

\frame{
\frametitle{Info (1 / 4)}
\begin{block}{The \ttfamily{Info}\sffamily{ Data Type}}\ttfamily
data Info = \bfseries{\textit{TyConI Dec}} \textbar  \\
~~~~~~~~~~~ DataConI Name Type Name Fixity \textbar \\ 	
~~~~~~~~~~~ VarI Name Type (Maybe Dec) Fixity \textbar	... \\
\end{block}

\begin{Example}\ttfamily\small
-- data Point = Pt Rational Rational \\
TyConI (DataD [] \\
~~~~~~~~~~~~~~Main.Point \\ 
~~~~~~~~~~~~~~[] \\
~~~~~~~~~~~~~~[NormalC Main.Pt\\ 
~~~~~~~~~~~~~~~~~~~~~~~[(NotStrict, ConT Rational), \\
~~~~~~~~~~~~~~~~~~~~~~~~(NotStrict, ConT Rational)]]  \\
~~~~~~~~~~~~~~[])
\end{Example}
}

\frame{
\frametitle{Info (2 / 4)}
\begin{block}{The \ttfamily{Info}\sffamily{ Data Type}}\ttfamily
data Info = TyConI Dec \textbar  \\
~~~~~~~~~~~ \textit{DataConI Name Type Name Fixity} \textbar \\ 	
~~~~~~~~~~~ VarI Name Type (Maybe Dec) Fixity \textbar ... \\
\end{block}

\begin{Example}\ttfamily
-- data Point = Pt Rational Rational \\
DataConI Main.Pt \\ 
~~~~~~~~~(AppT (AppT ArrowT (ConT Rational))  \\
~~~~~~~~~~~~~~~(AppT (AppT ArrowT (ConT Rational)) \\ 
~~~~~~~~~~~~~~~~~~~~~(ConT Main.Point)))  \\
~~~~~~~~~Main.Point \\
~~~~~~~~~(Fixity 9 InfixL) \\
\end{Example}
}

\frame{
\frametitle{Info (3 / 4)}
\begin{block}{The \ttfamily{Info}\sffamily{ Data Type}}\ttfamily
data Info = TyConI Dec \textbar  \\
~~~~~~~~~~~ DataConI Name Type Name Fixity \textbar \\ 	
~~~~~~~~~~~ \textit{VarI Name Type (Maybe Dec) Fixity} \textbar	... \\
\end{block}

\begin{Example}\ttfamily
-- x = 2.0 \\
VarI Main.x  \\
~~~~~(ConT Double) \\ 
~~~~~Nothing  \\
~~~~~(Fixity 9 InfixL)
\end{Example}
}

\frame{
\frametitle{Info (4 / 4)}
\begin{block}{The \ttfamily{Info}\sffamily{ Data Type}}\ttfamily
data Info = TyConI Dec \textbar  \\
~~~~~~~~~~~ DataConI Name Type Name Fixity \textbar \\ 	
~~~~~~~~~~~ \textit{VarI Name Type (Maybe Dec) Fixity} \textbar ... \\
\end{block}

\begin{Example}\ttfamily
-- foo x = x ++ "bar"\\
VarI Main.foo \\ 
~~~~~(AppT (AppT ArrowT (ConT GHC.Base.String)) \\
~~~~~~~~~~~(ConT GHC.Base.String)) \\
~~~~~Nothing \\ 
~~~~~(Fixity 9 InfixL)
\end{Example}
}





\section{Odds and Ends}
\subsection{Embedded DSLs and Metaprogramming}
\frame{
\frametitle{Embedded DSLs}
Template Haskell can be used to aid in the construction of embedded DSLs for Haskell.
\begin{itemize}
 \item Automatic creation of data types and functions
 \item Changing semantics of Haskell code
\end{itemize}
}

\frame{
\frametitle{Automatic creation of data types and functions (1/2)}
Reify data types and classes and automatically generate domain specific code.
\begin{itemize}
 \item Writing XML
 \item Network Transmission
 \item GUI Creation
 \item ...
\end{itemize}
}

\frame{
\frametitle{Automatic creation of data types and functions (2/2)}
At compile time you can generate data types and functions from domain specific declarations.
\begin{itemize}
\item Generate data type and functions from XML declarations (paper example)
\item Generate data types and functions for interacting with a database from a DB schema (ala Ruby on Rails)
\item Generate type safe functions for writing XML (tags and attributes) from declarative descriptions valid tags and attributes.
\end{itemize}
}

\frame{
\frametitle{Changing semantics of Haskell code}
Can use Quasi-quote to ensure type-safe syntactically valid Haskell is written, and then interpret it as you wish.
\begin{itemize}
 \item "Compiling" portions of Haskell programs to other programming languages such as JavaScript
 \item Randomized Linear Perturbations
\end{itemize}
}

\frame{
\frametitle{Randomized Linear Perturbations}
\begin{Example}\ttfamily
func :: Double -> Double -> Double -> IO Double \\
func x y z = \$(rlpTransform [\textbar ~x + 2 * y * z \textbar])
\end{Example}

\begin{block}{Spliced Body Is}\ttfamily\scriptsize
do zRand <- randomUniform01 \\
~~~yRand <- randomUniform01 \\
~~~xRand <- randomUniform01 \\
~~~return \\
~~~~~(let r = x + (2.0 * (y * z)) \\
~~~~~~in if not (r == (0\%1)) then r \\
~~~~~~~~~else let r = xRand + ((2.0 * (yRand * z)) + \\
~~~~~~~~~~~~~~~~~~~~~~~~~~~~~~~(2.0 * (y * zRand))) \\
~~~~~~~~~~~~~~in if not (r == (0\%1)) then r \\
~~~~~~~~~~~~~~~~~else let r = 2.0 * (yRand * zRand) \\
~~~~~~~~~~~~~~~~~~~~~~in if not (r == (0\%1)) then r else 0\%1)
\end{block}
}

\subsection{Scheme/Lisp Macros, MetaML, and Liskell}
\frame{
Metaprogramming is a common paradigm in functional programming, and is spreading to other scripting languages such as Python and Ruby.
\begin{itemize}
\item Lisp/Scheme have builtin procedures, very integral parts of the languages
\item MetaML is a metaprogramming extension to Standard ML
\item Liskell is another extension to Haskell to allow metaprogramming
\end{itemize}
}

\frame{
\frametitle{Lisp Programs are Lisp Data (1/2)}
\begin{example}\ttfamily
[d\textbar ~euclidDist (Pt x1 y1) (Pt x2 y2) = \\
~~~~~~let dx = (x1 - x2) \\
~~~~~~~~~~dy = (y1 - y2) \\
~~~~~~in sqrt( dx * dx + dy * dy ) \textbar]
\end{example}

\begin{block}{Haskell Syntax Tree}\ttfamily\scriptsize
~[FunD euclidDist [Clause [ConP Main.Pt [VarP x1\_0,VarP y1\_1],ConP Main.Pt [VarP x2\_2,VarP y2\_3]] (NormalB (LetE [ValD (VarP dx\_5) (NormalB (InfixE (Just (VarE x1\_0)) (VarE GHC.Num.-) (Just (VarE x2\_2)))) [],ValD (VarP dy\_4) (NormalB (InfixE (Just (VarE y1\_1)) (VarE GHC.Num.-) (Just (VarE y2\_3)))) []] (AppE (VarE GHC.Float.sqrt) (InfixE (Just (InfixE (Just (VarE dx\_5)) (VarE GHC.Num.*) (Just (VarE dx\_5)))) (VarE GHC.Num.+) (Just (InfixE (Just (VarE dy\_4)) (VarE GHC.Num.*) (Just (VarE dy\_4)))))))) []]]
\end{block}
}

\frame{
\frametitle{Lisp Programs are Lisp Data (2/2)}
\begin{example}\ttfamily
\textquoteleft(define (euclid-dist p1 p2)\\
~~~(let ((dx (- x1 x2))\\
~~~~~~~~~(dy (- y1 y2)))\\
~~~~~(sqrt (+ (* dx dx) (* dy dy)))))
\end{example}

\begin{block}{Scheme Syntax Tree}\ttfamily
(define (euclid-dist p1 p2) (let ((dx (- x1 x2)) (dy (- y1 y2))) (sqrt (+ (* dx dx) (* dy dy)))))
\end{block}
Scheme/Lisp syntax trees are are trivial to parse, construct, interpret, and composed entirely of normal scheme data types.
}

\frame{
\frametitle{Call Site Syntactic Baggage}
\begin{block}{Haskell}\ttfamily
func x y z = \$(rlpTransform [| (x + (2 * y * z)) |])
\end{block}

\begin{block}{Scheme}\ttfamily
(define (func x y z) (rlpTransform (+ x (* 2 y z))))
\end{block}
}

% \frame{
% \frametitle{Freedom (1/2)}
% Template Haskell, (\ttfamily{\$(...)}\sffamily{ and }\ttfamily{[\textbar ...\textbar]}\sffamily{), requires syntatically valid constructions. Scheme macros merely require parseablility. This allows for extension of the semantics of the language in many more intresting ways.}
% }

\frame{
\frametitle{Freedom}
Template Haskell, (\ttfamily{\$(...)}\sffamily{ and }\ttfamily{[\textbar ...\textbar]}\sffamily{), requires syntactically valid constructions. Scheme macros merely require parseablility. This allows for extension of the semantics of the language in many more intresting ways.}

\begin{block}{Haskell - requires syntactic correctness}\ttfamily
func x y z = \$(rlpTransform [| (x + (2 * y * z)) |])
\end{block}

\begin{block}{Scheme}\ttfamily
(define (func x y z) (rlpTransform x + 2 * y * z))
\end{block}

\begin{block}{Scheme}\ttfamily
(postfix (y z * 2 * x +))
\end{block}

% \begin{block}{Scheme}<3->\ttfamily
% (printf Error: \%s at line \%d msg line)
% \end{block}
}

\frame{
\frametitle{MetaML}
\begin{itemize}
 \item ML has similar syntax to Haskell, but is not lazy and is not purely functional
 \item MetaML also features a Quasi-Quoting feature, type-safety, static scoping.
 \item MetaML can build and execute code at run-time.
 \item Template Haskell can generate type unsafe programs but they won't compile, MetaML will only generate type safe programs (Necessary for run-time type safe execution) 
\end{itemize}
}

\frame{
\frametitle{Liskell}
http://clemens.endorphin.org/liskell - Syntax frontend over Haskell aimed at providing Lisp-like Syntax and metaprogramming facilities for Haskell programming.

\begin{block}{Liskell Example}\ttfamily
(defdata Point2DType (Point2D Rational Rational)) \\
~~\\
(define (distance2D (Point2D x1 y1) (Point2D x2 y2))\\
~~(let ((dx (- x1 x2))\\
~~~~~~~~(dy (- y1 y2)))\\
~~~~(sqrt (+ (* dx dx) (* dy dy)))))
\end{block}
}

\subsection{Questions}
\frame{
\frametitle{Questions}
\begin{center}
\LARGE{Questions...}
{\color{white} {\href{http://www.jmchilton.com/}{link}}}
\end{center}
}

\frame{
\frametitle{Three Levels of Abstraction}
Three levels of abstraction:
\begin{itemize}
 \item Ordinary algebraic data types represents Haskell syntax trees.
 \item Q monad wrappers over algebraic data types for generating fresh names and interacting with the compilers symbol table
 \item Quasi-Quote
\end{itemize}

\begin{Example}<2->\ttfamily
data Exp = AppE Exp Exp \textbar ~... \\
~\\
appE :: ExpQ -> ExpQ -> ExpQ \\
appE e1 e2 = do \{ a <- e1; b <- e2; return (AppE a b) \}
\end{Example}
}

\frame{
\frametitle{Typing Template Haskell}
Type checking and code execution must become interleaved.
\begin{Example}
\ttfamily{\$(printf "Error: \%s on line \%d.") msg line} \\
\end{Example}
When type checker encounters \$, it does the following
\begin{itemize}
 \item Type check body of splice
 \item Compile and execute body
 \item Splice in the resulting code
 \item Continue to type check result
\end{itemize}
}

\frame{
\frametitle{Cross-stage Persistence}
\begin{block}{}\ttfamily
module T( genSwap ) where \\
~~swap (a,b) = (b,a) \\
~~genSwap x = [\textbar swap x \textbar]
\end{block}

\begin{block}{}\ttfamily
module Foo where \\
~~import T(genSwap) \\\
~~swap = True \\
~~foo = \$(genSwap (4,5))
\end{block}

\begin{block}{swap}\sffamily
Quasi-Quote are lexically scoped, and \ttfamily{swap}\sffamily{ will be bound to the }\ttfamily{swap}\sffamily{ in scope at occurrence site in T even though it is not exported or in scope while in module }\ttfamily{Foo}.
\end{block}
}

\frame{
\frametitle{Lift}
\begin{block}{}\ttfamily
module T( genSwap ) where \\
~~swap (a,b) = (b,a) \\
~~genSwap x = [\textbar swap x \textbar]
\end{block}

\begin{block}{x}
x is a free variable and is inferred to be of class \ttfamily{Lift}.
\end{block}

\begin{block}{Lift}\ttfamily
class Lift t where \\
~~lift :: t -> ExpQ \\
instance Lift Int \\
~~lift n = litE (IntegerL t) \\
instance Lift (Lift a, Lift b) => Lift (a,b) where \\
~~lift(a,b) = tupE [lift a, lift b]
\end{block}
}


\end{document}
