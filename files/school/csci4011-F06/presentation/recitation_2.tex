% Handout Type
% \documentclass[handout]{beamer}
% Presentation Type
\documentclass{beamer}
\usepackage{amsmath,amsthm,ifthen}
\usepackage{johnscmds}
\usepackage{gastex}
\usepackage{beamerthemesplit}
\title{Recitation 2 - Deterministic Finite Automata}
\author{John Chilton}
\date{\today}

\begin{document}
\frame{\titlepage}

\section{Housekeeping}
\subsection{Today}
\frame{
We will be reviewing how to construct DFAs. We will start with some simple ones and use regular operations to construct more complicated ones.
}

\subsection{Last Week}
\frame{
\begin{itemize}
\item<1->Motivation and relevancy
\item<2->Use of Whiteboard/Chalkboard
\item<3->Slides
\end{itemize}
}

\section{Some Simple Examples}
\subsection{$R_1$}
\frame{
\begin{itemize}
\item<1->For alphabet $\Sigma = \{a,b\}$, $R_1 = \{w ~|~ w$ has an odd number of $b$'s$\}$. Find a DFA, $M_1 = (Q_1,\Sigma, \delta_1, q_1, F_1)$, that accepts $R_1$
\item<2->How many and which states are required?
\item<3->What should happen when $a$ is read?
\item<4->What should happen when $b$ is read?
\item<5->Label start and accept states.
\end{itemize}
}

\frame{
$$M_1 = (Q_1, \Sigma, \delta_1, q_1, F_1) = (\{q_{odd},q_{even}\}, \Sigma, \delta_1, q_{even}, \{q_{odd}\})$$
with $\delta_1$ as follows
\begin{center} 
\begin{tabular}{c|cc}
$\delta_1$   & $a$       & $b$    \\
\hline
$q_{odd}$    & $q_{odd}$ & $q_{even}$ \\ 
$q_{even}$   & $q_{even}$  & $q_{odd}$
\end{tabular}
\end{center}
}

% \subsection{$R_2$}
% \frame{
% \begin{itemize}
% \item<1->For alphabet $\Sigma = \{a,b\}$, $R_2 = \{w ~|~ w$ ends with an $b\}$. Find a DFA, $M_2 = (Q_2,\Sigma, \delta_1, q_1, F_1)$, that accepts $R_2$
% \item<2->How many and which states are required?
% \item<3->What should happen when $a$ is read?
% \item<4->What should happen when $b$ is read?
% \item<5->Label start and accept states.
% \end{itemize}
% }
% 
% \frame{
% $$M_2 = (Q_2, \Sigma, \delta_2, q_2, F_2) = (\{q_{nb}, q_{b}\}, \Sigma, \delta_2, q_{nb}, \{q_{b}\} )$$
% with $\delta_2$ as follows
% \begin{center} 
% \begin{tabular}{c|cc}
%           & $a$       & $b$    \\
% \hline
% $q_{nb}$ & $q_{nb}$ & $q_{b}$ \\ 
% $q_{b}$   & $q_{nb}$ & $q_{b}$
% \end{tabular}
% \end{center}
% }

\subsection{$R_2$}
\frame{
\begin{itemize}
\item<1->For alphabet $\Sigma = \{a,b\}$, $R_3 = \{w ~|~ w$ contains at least three $a$'s$\}$. Find a DFA, $M_2 = (Q_2,\Sigma, \delta_2, q_2, F_2)$, that accepts $R_2$
\item<2->How many and which states are required?
\item<3->What should happen when $a$ is read?
\item<4->What should happen when $b$ is read?
\item<5->Label start and accept states.
\end{itemize}
}

\frame{
\begin{align*}
M_2 
=& (Q_2, \Sigma, \delta_2, q_2, F_2) \\
=& (\{q_{20}, q_{21}, q_{22}, q_{23}\}, \Sigma, \delta_2, q_{20}, \{q_{23}\} )
\end{align*}
with $\delta_2$ as follows
\begin{center} 
\begin{tabular}{c|cc}
$\delta_2$  & $a$      & $b$    \\
\hline
$q_{20}$    & $q_{21}$ & $q_{20}$ \\ 
$q_{21}$    & $q_{22}$ & $q_{21}$ \\
$q_{22}$    & $q_{23}$ & $q_{22}$ \\ 
$q_{23}$    & $q_{23}$ & $q_{23}$
\end{tabular}
\end{center}
}

\subsection{$R_3$}
\frame{
\begin{itemize}
\item<1->For alphabet $\Sigma = \{a,b\}$, $R_3 = \{w ~|~ w$ starts with a $b$ and ends with an $a\}$. Find a DFA, $M_3 = (Q_3,\Sigma, \delta_3, q_3, F_3)$, that accepts $R_3$
\end{itemize}
}

\frame{
\begin{align*}
M_3 
=& (Q_3, \Sigma, \delta_3, q_3, F_3) \\
=& (\{q_{3start}, q_{3dead}, q_{3a}, q_{3b}\}, \Sigma, \delta_3, q_{3start}, \{q_{3a}\} )
\end{align*}
with $\delta_3$ as follows
\begin{center} 
\begin{tabular}{c|cc}
$\delta_3$   & $a$       & $b$    \\
\hline
$q_{3start}$ & $q_{3dead}$ & $q_{3b}$ \\ 
$q_{3dead}$  & $q_{3dead}$ & $q_{3dead}$ \\
$q_{3a}$     & $q_{3a}$    & $q_{3b}$ \\
$q_{3b}$     & $q_{3a}$    & $q_{3b}$ 
\end{tabular}
\end{center}
}

\section{Union, Intersection, and Complement}
\subsection{Union - $R_1 \cup R_2$}
\frame{
Given 
\begin{align*}
M_1 
=& (Q_1, \Sigma, \delta_1, q_1, F_1) \\
=& (\{q_{odd},q_{even}\}, \Sigma, \delta_1, q_{even}, \{q_{odd}\}) \\
M_2
=& (Q_2, \Sigma, \delta_2, q_2, F_2) \\
=& (\{q_{20}, q_{21}, q_{22}, q_{23}\}, \Sigma, \delta_2, q_{20}, \{q_{23}\} ) \\
\end{align*}
Construct DFA which accepts $R_1 \cup R_2 = (Q_\cup, \Sigma, \delta_\cup, q_0, F_\cup)$.
}

\frame{
$Q_\cup$
\pause
\begin{align*}
Q_\cup =&  Q_1 \times Q_2 \\ 
=&\{(q_{odd}, q_{20}),(q_{odd}, q_{21}), (q_{odd}, q_{22}), (q_{odd}, q_{23}), \\
 &(q_{even}, q_{20}),(q_{even}, q_{21}), (q_{even}, q_{22}), (q_{even}, q_{23}) \}
\end{align*}
}

\frame{
  \begin{center}
    \unitlength=4pt
    \begin{picture}(60, 24)(0,-2)
    \gasset{Nw=16,Nh=5,Nmr=2.5,curvedepth=0}
    \thinlines
    \node(A1)(0,12){$(q_{odd}, q_{20})$}
    \node(A2)(20,12){$(q_{odd}, q_{21})$}
    \node(A3)(40,12){$(q_{odd}, q_{22})$}
    \node(A4)(60,12){$(q_{odd}, q_{23})$}
    \node(A5)(00,00){$(q_{even}, q_{20})$}
    \node(A6)(20,00){$(q_{even}, q_{21})$}
    \node(A7)(40,00){$(q_{even}, q_{22})$}
    \node(A8)(60,00){$(q_{even}, q_{23})$}
    \end{picture}
  \end{center}
}

\frame{ 
$q_0$ and $F_\cup$
\pause
\begin{align*}
q_0  
=&  (q_1, q_2) \\ 
=&  (q_{even}, q_{20}) \\
\end{align*}
\pause
\begin{align*}
F_\cup 
=& \{(r_1, r_2) ~|~ r_1 \in F_1 \text{ or } r_2 \in F_2\} \\
=& \{(q_{odd},q_{20}), (q_{odd},q_{21}), (q_{odd},q_{22}), (q_{odd},q_{23}), (q_{even},q_{23})\}
\end{align*}
}

\frame{
  \begin{center}
    \unitlength=4pt
    \begin{picture}(60, 24)(0,-2)
    \gasset{Nw=16,Nh=5,Nmr=2.5,curvedepth=0}
    \thinlines
    \node[Nmarks=r](A1)(0,12){$(q_{odd}, q_{20})$}
    \node[Nmarks=r](A2)(20,12){$(q_{odd}, q_{21})$}
    \node[Nmarks=r](A3)(40,12){$(q_{odd}, q_{22})$}
    \node[Nmarks=r](A4)(60,12){$(q_{odd}, q_{23})$}
    \node[Nmarks=i](A5)(00,0){$(q_{even}, q_{20})$}
    \node(A6)(20,0){$(q_{even}, q_{21})$}
    \node(A7)(40,0){$(q_{even}, q_{22})$}
    \node[Nmarks=r](A8)(60,0){$(q_{even}, q_{23})$}
    \end{picture}
  \end{center}
}

\frame{

\begin{center}
\begin{tabular}{ccc}
\begin{tabular}{c|cc}
$\delta_1$             & $a$       & $b$    \\
\hline
$q_{odd}$    & $q_{odd}$ & $q_{even}$ \\ 
$q_{even}$   & $q_{even}$  & $q_{odd}$
\end{tabular}
&~~~~~~~~~~~~~~~~&
\begin{tabular}{c|cc}
$\delta_2$         & $a$      & $b$    \\
\hline
$q_{20}$ & $q_{21}$ & $q_{20}$ \\ 
$q_{21}$ & $q_{22}$ & $q_{21}$ \\
$q_{22}$ & $q_{23}$ & $q_{22}$ \\ 
$q_{23}$ & $q_{23}$ & $q_{23}$
\end{tabular}
\end{tabular}
\pause
\bigskip
\begin{tabular}{c|cc}
$\delta_\cup$         & $a$      & $b$    \\
\hline
$(q_{odd}, q_{20})$  & $(q_{odd},q_{21})$ & $(q_{even},q_{20})$ \\ 
$(q_{odd}, q_{21})$  & $(q_{odd},q_{22})$ & $(q_{even},q_{21})$ \\
$(q_{odd}, q_{22})$  & $(q_{odd},q_{23})$ & $(q_{even},q_{22})$ \\ 
$(q_{odd}, q_{23})$  & $(q_{odd},q_{23})$ & $(q_{even},q_{23})$ \\
$(q_{even}, q_{20})$ & $(q_{even},q_{21})$ & $(q_{odd},q_{20})$ \\ 
$(q_{even}, q_{21})$ & $(q_{even},q_{22})$ & $(q_{odd},q_{21})$ \\
$(q_{even}, q_{22})$ & $(q_{even},q_{23})$ & $(q_{odd},q_{22})$ \\ 
$(q_{even}, q_{23})$ & $(q_{even},q_{23})$ & $(q_{odd},q_{23})$ 
\end{tabular}
\end{center}
}

\frame{
  \begin{center}
    \unitlength=4pt
    \begin{picture}(60, 24)(0,-2)
    \gasset{Nw=16,Nh=5,Nmr=2.5,curvedepth=0}
    \thinlines
    \node[Nmarks=r](A1)(0,12){$(q_{odd}, q_{20})$}
    \node[Nmarks=r](A2)(20,12){$(q_{odd}, q_{21})$}
    \node[Nmarks=r](A3)(40,12){$(q_{odd}, q_{22})$}
    \node[Nmarks=r](A4)(60,12){$(q_{odd}, q_{23})$}
    \node[Nmarks=i](A5)(00,0){$(q_{even}, q_{20})$}
    \node(A6)(20,0){$(q_{even}, q_{21})$}
    \node(A7)(40,0){$(q_{even}, q_{22})$}
    \node[Nmarks=r](A8)(60,0){$(q_{even}, q_{23})$}

    \drawedge(A1,A2){$a$}
    \drawedge[sxo=-1,exo=-1,ELside=r](A1,A5){$b$}
    \drawedge(A2,A3){$a$}
    \drawedge[sxo=-1,exo=-1,ELside=r](A2,A6){$b$}
    \drawedge(A3,A4){$a$}
    \drawedge[sxo=-1,exo=-1,ELside=r](A3,A7){$b$}
    \drawloop(A4){$a$}
    \drawedge[sxo=-1,exo=-1,ELside=r](A4,A8){$b$}

    \drawedge(A5,A6){$a$}
    \drawedge[sxo=1,exo=1,ELside=r](A5,A1){$b$}
    \drawedge(A6,A7){$a$}
    \drawedge[sxo=1,exo=1,ELside=r](A6,A2){$b$}
    \drawedge(A7,A8){$a$}
    \drawedge[sxo=1,exo=1,ELside=r](A7,A3){$b$}
    \drawloop[loopangle=270](A8){$a$}
    \drawedge[sxo=1,exo=1,ELside=r](A8,A4){$b$}
    \end{picture}
  \end{center}
}

\subsection{Intersection - $R_1 \cap R_2$}
\frame{
Intersection. 
\pause
$F_\cap = \{(r_1, r_2) ~|~ r_1 \in F_1 \text{ and } r_2 \in F_2\}$
\pause
  \begin{center}
    \unitlength=4pt
    \begin{picture}(60, 24)(0,-2)
    \gasset{Nw=16,Nh=5,Nmr=2.5,curvedepth=0}
    \thinlines
    \node(A1)(0,12){$(q_{odd}, q_{20})$}
    \node(A2)(20,12){$(q_{odd}, q_{21})$}
    \node(A3)(40,12){$(q_{odd}, q_{22})$}
    \node[Nmarks=r](A4)(60,12){$(q_{odd}, q_{23})$}
    \node[Nmarks=i](A5)(00,0){$(q_{even}, q_{20})$}
    \node(A6)(20,0){$(q_{even}, q_{21})$}
    \node(A7)(40,0){$(q_{even}, q_{22})$}
    \node(A8)(60,0){$(q_{even}, q_{23})$}

    \drawedge(A1,A2){$a$}
    \drawedge[sxo=-1,exo=-1,ELside=r](A1,A5){$b$}
    \drawedge(A2,A3){$a$}
    \drawedge[sxo=-1,exo=-1,ELside=r](A2,A6){$b$}
    \drawedge(A3,A4){$a$}
    \drawedge[sxo=-1,exo=-1,ELside=r](A3,A7){$b$}
    \drawloop(A4){$a$}
    \drawedge[sxo=-1,exo=-1,ELside=r](A4,A8){$b$}

    \drawedge(A5,A6){$a$}
    \drawedge[sxo=1,exo=1,ELside=r](A5,A1){$b$}
    \drawedge(A6,A7){$a$}
    \drawedge[sxo=1,exo=1,ELside=r](A6,A2){$b$}
    \drawedge(A7,A8){$a$}
    \drawedge[sxo=1,exo=1,ELside=r](A7,A3){$b$}
    \drawloop[loopangle=270](A8){$a$}
    \drawedge[sxo=1,exo=1,ELside=r](A8,A4){$b$}
    \end{picture}
  \end{center}
}

%TODO
\subsection{Via NFAs}
\frame{
Producing an NFA that is that accepts the union of two regular languages given by DFAs (or NFAs) is even easier.
\pause
  \begin{center}
    \unitlength=4pt
    \begin{picture}(54, 30)(0,-10)
    \gasset{Nw=10,Nh=5,Nmr=2.5,curvedepth=0}
    \thinlines
    \node[Nmarks=i](A0)(0,10){$q_{new}$}
    \node[Nmarks=r](A1)(48,10){$q_{odd}$}
    \node(A2)(32,10){$q_{even}$}
    \node(A5)(0,0){$q_{20}$}
    \node(A6)(16,0){$q_{21}$}
    \node(A7)(32,0){$q_{22}$}
    \node[Nmarks=r](A8)(48,0){$q_{23}$}

    \drawedge(A0,A2){$\epsilon$}
    \drawedge(A0,A5){$\epsilon$}

    \drawedge[syo=1,eyo=1,ELside=r](A1,A2){$b$}
    \drawedge[syo=-1,eyo=-1,ELside=r](A2,A1){$b$}
    \drawloop[loopangle=90](A1){$a$}
    \drawloop[loopangle=90](A2){$a$}


    \drawedge(A5,A6){$a$}
    \drawedge(A6,A7){$a$}
    \drawedge(A7,A8){$a$}
    \drawloop[loopangle=270](A5){$b$}
    \drawloop[loopangle=270](A6){$b$}
    \drawloop[loopangle=270](A7){$b$}
    \drawloop[loopangle=270](A8){$a,b$}

    \end{picture}
  \end{center}
\pause
How about intersection?
}

\subsection{Complement - $\bar{R_3}$}
\frame{ 
\begin{itemize}
\item<1->Reminder: $R_3 = \{w ~|~ w$ starts with a $b$ and ends with an $a\}$
\item<2->What is the complement of $R_3$
\item<3->Design a DFA which accepts $\bar{R_3}$.
\item<4->Does the same trick work for NFAs?
\end{itemize}
}



\section{Star and Concatenation}
\subsection{Star and Concatenation}
\frame{
\begin{itemize}
\item<1->Easy, straightforward methods given to compute union, intersection, and complement of regular languages
\item<2->Star and concatenation are are slightly more complicated general...
\item<3->$R_4 = \{w ~|~ w \text{ contains exactly 2 } a$'s$\}$ 
\item<3->$R_5 = \{w ~|~ w \text{ contains exactly 2 } b$'s$\}$
\item<3->$R_4\circ R_5$ - consider \textbf{$baabbab$} and \textbf{$baabab$}
\item<4->$(R_4\cup R_5)^\ast$ - difficult for same reason
\end{itemize}
}

\subsection{Some Examples}
\frame{
\begin{itemize}
\item<1->Consider $R_6 = baa^\ast$.
\item<2->Design a DFA which accepts $R_6^\ast$
\item<3->Design a DFA which accepts $R_6\circ R_1$
\end{itemize}
}


\end{document}


%\begin{VCPicture}{(-1,1)(4,1)}
%\State[p]{(0,0)}{A} \State[q]{(3,0)}{B}
%\Initial{A} \Final{B}
%\EdgeL{A}{B}{a}
%\end{VCPicture}

%\begin{VCPicture}{(-1,1)(4,1)}
%\State[p]{(0,0)}{A} \FinalState[q]{(3,0)}{B}
%\Initial{A}
%\EdgeL{A}{B}{a}
%\end{VCPicture}
