% Handout Type
\documentclass[handout]{beamer}
% Presentation Type
%\documentclass{beamer}
\usepackage{amsmath,amsthm,ifthen}
\usepackage{johnscmds}
\usepackage{gastex}
\usepackage{beamerthemesplit}
\title{Recitation 3 - Homework 1}
\author{John Chilton}
\date{\today}

\begin{document}

\frame{\titlepage}

\section{Housekeeping}
\subsection{Today}
\frame{
Questions and examples related to homework 1.


Example Conversions: Regular Expression to/from NFAs and NFAs to DFAs.
}

\subsection{Questions?}
\frame{
\begin{center}
\LARGE{Questions?}
\end{center}
}

% \section{}
% \subsection{}
% \frame{
% \begin{center}
% \unitlength=4pt
% \begin{picture}(60, 24)(0,-2)
%     \gasset{Nw=10,Nh=5,Nmr=2.5,curvedepth=0}
%     \thinlines
%     \node[Nmarks=i](A1)(0,6){$q_{1}$}
%     \node(A2)(14,12){$q_{2}$}
%     \node[Nmarks=r](A3)(28,12){$q_{3}$}
%     \node(A4)(14,0){$q_{4}$}
%     \drawedge(A1,A2){$a$}
%     \drawedge(A1,A4){$b$}
%     \drawedge[syo=-1,eyo=-1,ELside=r](A2,A3){$b$}
%     \drawedge[syo=1,eyo=1,ELside=r](A3,A2){$a$}
%     \drawloop[loopangle=0](A3){$b$}
%     \drawloop[loopangle=0](A4){$a,b$}
% \end{picture}
% \end{center}
% }

\section{Problems}
\subsection{1.15}
\frame{
\pause
\begin{center}
% \unitlength=4pt
% \begin{picture}(0, 0)(10,-12)
%     \gasset{Nw=6,Nh=5,Nmr=2.5,curvedepth=0}
%     \thinlines
%     \node[Nmarks=i](A1)(0,0){$1$}
%     \node(A2)(12,0){$2$}
%     \node[Nmarks=r](A3)(24,0){$3$}
%     \drawedge(A1,A2){$1$}
%     \drawedge(A2,A3){$0$}
% %     \drawedge[syo=-1,eyo=-1,ELside=r](A2,A3){$b$}
% %     \drawedge[syo=1,eyo=1,ELside=r](A3,A2){$a$}
%     \drawloop[loopangle=0](A3){$1$}
%     \drawloop[loopangle=270](A2){$1$}
% \end{picture}
\begin{tabular}{c|ccc}
$\delta_1$   & $0$      & $1$     & $\epsilon$ \\
\hline 
$1$        & $\phi$   & $\{2\}$ & $\phi$       \\ 
$2$        & $\{3\}$  & $\{2\}$ & $\phi$       \\
$3$        & $\phi$   & $\{3\}$ & $\phi$
\end{tabular}
\pause
$$\Downarrow\Downarrow\Downarrow$$

\begin{tabular}{c|ccc}
\textcolor{red}{$\delta$}   & $0$      & $1$     & $\epsilon$ \\
\hline 
$1$        & $\phi$   & $\{2\}$ & $\phi$       \\ 
$2$        & $\{3\}$  & $\{2\}$ & $\phi$       \\
$3$        & $\phi$   & $\{3\}$ & $\phi$ \textcolor{red}{$\cup \{1\} = \{1\}$}
\end{tabular}
\end{center}
}

\frame{
\begin{itemize}
\item This construction works for this example.
\item Problem 1.15 ask for an example where this doesn't work.
\end{itemize}
}

\subsection{1.31}
\frame{
\begin{itemize}
\item If language $A$ is regular, prove $A^R$ is regular.
\item Given DFA $M$ which recognizes $A$, construct DFA or NFA $M_{1}$ which recognizes $A^{R}$.
\item Remember to prove both ways:
\begin{itemize}
\item Given $w \in A$, show $M_1$ accepts $w^R$, and given $w \notin A$, show $w^R$ not accepted by $M_1$.
\item -or- Given $w \in A$, show $M_1$ accepts $w^R$, and given $w$ accepted by $M_1$ show $w^R \in A$.
\end{itemize}
\end{itemize}
}

\section{Convert Methods}
\subsection{Regular Expression $\rightarrow$ NFA}
\frame{
$$1 \cup (01^*0) \pause =  (1)\cup(0(((1)^*)0))$$
}
\subsection{NFA $\rightarrow$ Regular Expression}
\frame{
\begin{center}
\unitlength=4pt
\begin{picture}(10, 24)(0,-12)
    \gasset{Nw=6,Nh=5,Nmr=2.5,curvedepth=0}
    \thinlines
    \node(A1)(0,12){$q_{i}$}
    \node(A2)(12,12){$q_{j}$}
    \node(A3)(6,0){$q_{r}$}
    \drawedge(A1,A2){$R_4$}
    \drawedge[ELside=r](A1,A3){$R_1$}
    \drawedge[ELside=r](A3,A2){$R_3$}
    \drawloop[loopangle=270](A3){$R_2$}
\end{picture}

Remove $R_1$ and $R_3$ and replace $R_4$ with $R_1R_2^{*}R_3 \cup R_4$.
\end{center}
}

\subsection{NFA $\rightarrow$ DFA}
\frame{
\begin{center}
\unitlength=4pt
\begin{picture}(60, 24)(0,-2)
    \gasset{Nw=6,Nh=5,Nmr=2.5,curvedepth=0}
    \thinlines
    \node[Nmarks=i](A1)(0,6){$1$}
    \node(A2)(14,12){$2$}
    \node[Nmarks=r](A3)(28,12){$3$}
    \node(A4)(14,0){$4$}
    \node(A5)(28,0){$5$}
    \drawedge(A1,A2){$b$}
    \drawedge(A1,A4){$a,\epsilon$}
    \drawedge(A2,A3){$b$}
    \drawedge[syo=-1,eyo=-1,ELside=r](A4,A5){$a$}
    \drawedge[syo=1,eyo=1,ELside=r](A5,A4){$a$}
    \drawloop[loopangle=90](A2){$a,b$}
    \drawloop[loopangle=270](A4){$b$}
    \drawloop[loopangle=270](A5){$b$}
\end{picture}
\end{center}
}

\frame{
\begin{center}
\unitlength=4pt
\begin{picture}(85, 24)(0,-2)
    \gasset{Nw=12,Nh=5,Nmr=2.5,curvedepth=0}
    \thinlines

    \node[Nmarks=i](A1)(20,0){$\{1,4\}$}
    \node(A2)(50,0){$\{4,5\}$}
    \node(A3)(20,12){$\{2,4\}$}
    \node(A4)(50,12){$\{2,5\}$}
    \node[Nmarks=r](A5)(20,24){$\{2,3,4\}$}
    \node[Nmarks=r](A6)(50,24){$\{2,3,5\}$}
    \drawedge(A1,A2){$a$}
    \drawedge(A1,A3){$b$}
    \drawedge[syo=1,eyo=1](A3,A4){$a$}
    \drawedge(A3,A5){$b$}
    \drawedge(A5,A4){$a$}
    \drawedge(A6,A3){$a$}
    \drawedge[syo=-1,eyo=-1](A4,A3){$a$}
    \drawedge(A4,A6){$b$}
    \drawloop[loopangle=90](A5){$b$}
    \drawloop[loopangle=90](A6){$b$}
    \drawloop[loopangle=270](A2){$a,b$}
\end{picture}
\end{center}
}





% \frame{
% Let $M_1=(Q, \Sigma, \delta, q_0, F)$ be a DFA that recoginizes $A$. The following NFA recoginizes $M=(Q, \Sigma, \delta', q_0, F)$ recoginizes $NOPREFIX(A)$, where $\delta'$ is defined as follows:
% 
% \begin{align*}
% \delta'(r,a) = 
% \begin{cases}
% \{\delta(r,a)\}& \text{if $r \notin F$} \\
% \{\} & \text{if $r \in F$}
% \end{cases}
% \end{align*}
% }
% 
% \frame{
% If $w\notin A$, then no path exists in $M_1$ to a final state on input $w$, so clearly $w\notin L(M)$ because we have only removed possible transitions.
% 
% If $w\in A$, then there exists a sequence of states $r_0,r_1,\hdots,r_n$ s.t. 
% \begin{itemize}
% \item $r_0 = q_0$ 
% \item For $i=1,2,\hdots,n-1$, $r_{i+1} = \delta(r_i, w_{i+1})$
% \item $r_n \in F$
% \end{itemize}
% }

\end{document}


%\begin{VCPicture}{(-1,1)(4,1)}
%\State[p]{(0,0)}{A} \State[q]{(3,0)}{B}
%\Initial{A} \Final{B}
%\EdgeL{A}{B}{a}
%\end{VCPicture}

%\begin{VCPicture}{(-1,1)(4,1)}
%\State[p]{(0,0)}{A} \FinalState[q]{(3,0)}{B}
%\Initial{A}
%\EdgeL{A}{B}{a}
%\end{VCPicture}
