% Handout Type
\documentclass[handout]{beamer}
% Presentation Type
%\documentclass{beamer}
\usepackage{amsmath,amsthm,ifthen}
\usepackage{johnscmds}
\usepackage{beamerthemesplit}
\title{Recitation 1 - Review and the Cardinality of Sets}
\author{John Chilton}
\date{\today}

\begin{document}
\frame{\titlepage}
%\section[Outline]{}
%\frame{\tableofcontents}

\section{Housekeeping}
\subsection{Contact Info}
\frame{
\begin{itemize}
\item John Chilton
\item Office Hours: Monday 12:30-1:30 and Wednesday 11:30-12:30 in EE/CS 2-209
\item E-Mail: chilton at cs dot umn dot edu
\item Phone: 612-226-9223
\item Skype: jmchilton
\item IM Contact Info 
\begin{itemize}
\item AIM: johnny4413
\item Jabber/Google Talk: jmchilton at gmail dot com
\item Yahoo Messanger: chil0060
\end{itemize}
\end{itemize}
}

\subsection{Recitations}
\frame{
\begin{itemize}
\item<1-> A few words about slides...
\item<2-> Note cards - at the end of recitation write down one thing that presented clearly and one thing that wasn't. Feel free to not write anything.
\item<3-> Come with questions and ask questions!
\end{itemize}
}

\section{Functions and Cardinality}
\subsection{Motivation}
\frame{
\begin{itemize}
\item<1-> Cardinality of set $A$, represented as $|A|$, is the "number" of elements in $A$.
\item<2-> Fairly straight forward for finite sets, try one $\{1,5,9\}$
\end{itemize}
}

\subsection{Functions}
\frame{
\begin{itemize}
\item<1-> A \emph{function} or \emph{mapping} is an object that specifies an input-output relationship.
\item<2-> Notation: $f : D \rightarrow R$
\item<3-> Some intresting functions: 
\begin{itemize}
\item<4-> \emph{One-to-one} : No two elements in the domain map to the same element in the range
\item<5-> \emph{Onto} : Each element in the range set has an element from the domain mapped to it
\item<6-> \emph{Bijection} : A function that is both one-to-one and onto
\end{itemize}
\end{itemize}
}

\subsection{Equinumerosity}
\frame{
\emph{Equinumerosity} : Sets $A$ and $B$ are equinumerous if there exists a bijection 
$$ f: A \rightarrow B $$
\pause An equivalence relation that describes intuitive concept of two sets being the same "size". Works for finite and infinite sets.
}

\subsection{Cardinality}
\frame{
The cardinality of set $A$ is 
\begin{itemize}
\item<1-> $n$ if $A$ is equinumerous with $\{1, 2, \hdots, n\}$
\item<2-> \emph{Countably Infinite} if $A$ is equinumerous with $\mathcal{N}$, we represent this symbolically as $|A| = |\mathcal{N}| = \aleph_0$.
\item<3-> \emph{Uncountably Infinite} if $A$ is infinite and not equinumerous with $\mathcal{N}$, all such sets are "larger" than $\mathcal{N}$
\end{itemize}
}


\section{Examples}
\subsection{Examples of Countably Infinite Sets}
\frame{
\begin{itemize}
\item $A = \{2,3,4,\hdots\}$ \pause \textcolor{red}{$$f(n) = n-2$$}
\item \pause Set of even natural numbers \pause \textcolor{red}{$$f(n) = n/2$$}
\item \pause Set of integers $\mathcal{Z}$ \pause 
\textcolor{red}{$$f(n) = \begin{cases} 2|n| & n \ge 0 \\ 2|n| - 1 & n < 0\end{cases}$$}
\end{itemize}
}

\subsection{A more involved example - $\mathcal{N} \times \mathcal{N}$}
\frame{
\begin{itemize}
\item<1-> Question: Is the set $\mathcal{N} \times \mathcal{N}$ ($\{(i,j) | i,j \in \mathcal{N}\}$) countable?
\item<2-> Answer: Yes, a suitable bijection can be shown to be
$$ f(i,j) = \frac{(i+j)*(i+j+1)}{2} + i $$ 
\item<3-> Proof: Show $f$ is a bijection mapping $\mathcal{N} \times \mathcal{N}$ to $\mathcal{N}$. Show it is one-to-one and onto.
\end{itemize}
}

\subsection{An uncountable set}
\frame{
\begin{itemize}
\item<1-> Cantor's Theorem: The power set of any set is "larger" than the set itself.
\item<2-> Consider $\mathit{P}(\mathcal{N})$
\item<3-> Assume a bijection $f : \mathit{P}(\mathcal{N}) \rightarrow \mathcal{N} $ exists and show a contradiction.
\end{itemize}
}

\subsection{Diagonalization}
\frame{
\begin{itemize}
\item<1-> For $i = 0,1,2,\hdots$ let $A_i$ be the element of $\mathit{P}(\mathcal{N})$ that maps to $i$.
\item<2-> Now consider the set $S = \{ i \in \mathcal{N} ~|~ i \notin A_i\}$. 
\item<3-> $S$ is a member of $\mathit{P}(\mathcal{N})$ but does not map to any element in $\mathcal{N}$ - a contradiction.
\end{itemize}
}

\end{document}


%\begin{VCPicture}{(-1,1)(4,1)}
%\State[p]{(0,0)}{A} \State[q]{(3,0)}{B}
%\Initial{A} \Final{B}
%\EdgeL{A}{B}{a}
%\end{VCPicture}

%\begin{VCPicture}{(-1,1)(4,1)}
%\State[p]{(0,0)}{A} \FinalState[q]{(3,0)}{B}
%\Initial{A}
%\EdgeL{A}{B}{a}
%\end{VCPicture}


\end{document}
