% Handout Type
\documentclass[handout]{beamer}
% Presentation Type
%\documentclass{beamer}
\usepackage{amsmath,amsthm,ifthen}
\usepackage{johnscmds}
\usepackage{gastex}
\usepackage{beamerthemesplit}
\title{Recitation 12 - Homework 5}
\author{John Chilton}
\date{\today}

\begin{document}
\frame{\titlepage}

\section{Housekeeping}
\subsection{Today}
\frame{
\begin{itemize}
\item Homework 5 Problems
\item Reduction
\end{itemize}
}

% \section{Homework 5 }
% \subsection{Problems 1 and 2}
% 
% \frame{
% Problem 1 (Exercise 4.6). Let $\mathcal{B}$ be the set of all infinite sequences of $0$s and $1$s.  Show $\mathcal{B}$ is not countable using a proof by diagnolization.
% }
% 
% \frame{
% Assume some $f$ exists that is a one-to-one and onto mapping between $\mathcal{B}$ and $\mathcal{N}$.
% \begin{center} 
% \begin{tabular}{c|c}
% $n$   & $f(n)$    \\
% \hline
% $1$ & $\textbf{\underline{0}}01010101...$  \\ 
% $2$ & $1\textbf{\underline{0}}0101001...$  \\
% $3$ & $10\textbf{\underline{1}}111011...$  \\
% $4$ & $001\textbf{\underline{0}}01010...$  \\
% $5$ & $1010\textbf{\underline{1}}1101...$  \\
% $6$ & $11100\textbf{\underline{1}}010...$  \\
% \vdots & \vdots
% \end{tabular}
% \end{center}
% Construct a new infinite sequence $b$ such that if the $i^{th}$ element of $f(i)$ if 0 then the $i^{th}$ element of $b$ should be 1, and 0 otherwise.
% }
% 
% 
% \frame{
% Problem 2. (Exercise 4.7) Show the following set $T$ is countable.
% $$T = \{(i,j,k) ~|~ i,j,k \in \mathcal{N}\}$$
% }
% 
% \subsection{Problems 3-7}
% \frame{
% Problem 3. (Problem 4.10)
% $$INFINITE_{PDA} = \{M~|~ M \text{ is a PDA and $L(M)$ is an infinite language }\} $$
% Show $INFINITE_{PDA}$ is decidable.
% 
% \begin{tabular}{ll}
% $T$=& On input $M$ where $M$ is a PDA. \\ 
% & 1. Calculate pumping length $p$ of $L(M)$ from an equivlent CFG. \\
% & 2. Construct a DFA, $A$, which accepts a given $w$ iff $|w| \ge p$.\\
% & 3. Construct a PDA, $B$, which recoginizes $L(M) \cap L(A)$ \\
% & 4. Run $E_{CFG}$ with a CFG created from $B$. \\
% & 5. If it rejects, \emph{accept}, else \emph{reject}. \\
% \end{tabular}
% }
% 
% \frame{
% Problem 4. (Problem 4.25)
% $$E = \{M~|~M\text{ is a DFA that accepts some string with more 0s than 1s}\}$$
% Show $E$ is decidable.
% ~\\
% \pause
% We will need the following CFG G which generates the set of all strings that contain more 0s than 1s.
% ~\\
% \begin{align*}
% S &\rightarrow 0T ~|~ 0S ~|~ 1SS\\
% T &\rightarrow 0U ~|~ 1V ~|~ \epsilon \\
% U &\rightarrow 0UU ~|~ 1T \\
% V &\rightarrow 0T ~|~ 1VV
% \end{align*}
% }
% 
% \frame{
% \begin{tabular}{ll}
% $T$=& On input $M$ where $M$ is a DFA. \\ 
% & 1. Construct $G$ described above and convert it into PDA $P$. \\
% & 2. Construct a PDA, $B$, which recoginizes $L(M) \cap L(P)$ \\
% & 3. Run $E_{CFG}$ with a CFG created from $B$. \\
% & 4. If it rejected, \emph{accept}, else \emph{reject}. \\
% \end{tabular}
% }
% 
% \frame{
% Problem 5. (Problem 4.26)
% $C = \{(G,x)~|~G $ is CFG that generates some string $w$ where $x$ is a substring  of $w\}$
% \pause
% ~\\
% ~\\
% \begin{tabular}{ll}
% $T$=& On input $<G,x>$ where $G$ is a CFG. \\ 
% & 1. Construct a DFA $M$ from reg. exp. $\Sigma^*x\Sigma^*$. \\
% & 2. Construct a PDA $B$ which recognizes $L(M) \cap L(G)$ \\
% & 3. Run $E_{CFG}$ with a CFG created from $B$. \\
% & 4. If it rejected, \emph{accept}, else \emph{reject}. \\
% \end{tabular}
% }
% 
% \frame{
% Problem 6.
% $A$ is Turing recognizable language of deciding Turing machines $\{M_1, M_2, M_3,\hdots\}$. Need to show there exists a Turing decidable language $D$ such that $L(D) \ne M_i$ for all $i$.
% \pause
% \begin{center} 
% \begin{tabular}{c||c|c}
% $n$   & $w$ & $E$     \\
% \hline
% $1$ &  $\epsilon$ & $M_{7893}$\\ 
% $2$ &  0  & $M_{7}$           \\
% $3$ &  1  & $M_{3234}$        \\
% $4$ &  00 & $M_{7}$           \\
% $5$ &  01 & $M_{13}$          \\
% $6$ &  10 & $M_{1}$           \\
% $7$ &  11 & $M_{12}$          \\
% $8$ & 000 & $M_{3233}$        \\
% \vdots & \vdots & \vdots
% \end{tabular}
% \end{center}
% }
% 
% \frame{
% \begin{tabular}{ll}
% $T$=& On input $w$. \\
% & 1. Determine index $i$ of $w$. \\
% & 2. Run enumerator $E$ until $i^{th}$ TM printed. \\
% & 3. Run $w$ on this $i^{th}$ machine. \\
% & 4. If it rejects, \emph{accept}, else \emph{reject}. \\
% \end{tabular}
% \pause
% ~\\
% ~\\
% Since $E$ will eventually print any given $M_i$, $T$ will decide a different language than each $M_i$, so $D = L(T)$ is diffent than each $L(M_i)$. 
% }
% 
% \frame{
% \begin{center}
% Problem 7. $C$ is Turing recognizable iff it can be expressed as $\{x ~|~ \text{for some } y, (x,y) \in D\}$ for some decidable language $D$.
% \end{center}
% }
% 
% \frame{
% Problem 7. $C$ is Turing recognizable iff it can be expressed as $\{x ~|~ \text{for some } y , (x,y) \in D\}$ for some decidable language $D$.
% ~\\
% ~\\
% Assume there exists a $D$ such that:
% $$C = \{x ~|~ \text{for some } y, (x,y) \in D\}$$. 
% Need to show $C$ is Turing recognizable. 
% }
% 
% \frame{
% $D$ is decidable, so some deciding TM $M$ decides it. We will use $M$ to build a TM, $T$, that recognizes $\{x ~|~ \text{for some } y, (x,y) \in D\}$. Then $C = L(T)$ is the desired Turing recognizable language.
% ~\\
% ~\\
% \pause
% \begin{tabular}{ll}
% $T$=& On input $x$. \\ 
% & 1. Foreach $y$ in $\epsilon, 0,1,00,01,10,11,000,\hdots$ \\
% & 2. ~~~~ Run $(x,y)$ on $M$, \emph{accept} if $M$ does. \\
% \end{tabular}
% }
% 
% \frame{
% Problem 7. 
% $C$ is Turing recognizable iff it can be expressed as $\{x ~|~ \text{for some } y, (x,y) \in D\}$ for some decidable language $D$.
% ~\\
% ~\\
% Assume $C$ is Turing recognizable. Show $C=\{x ~|~ \text{for some } y, (x,y)\in D \}$ for some Turing decidable $D$. 
% ~\\
% ~\\
% \pause
% $C$ is Turing recognizable some TM, $M$, recognizes it. For each $x\in C$, $M$ will accept it in some number of steps, say $y$. Let $D = \{(x,y) ~|~ M \text{ accpets } x \text{ in } y \text{ steps }\}$.
% \begin{itemize}
%  \item $D$ is decidable.
%  \item $C = \{x ~|~ \exists y, (x,y) \in D\}$
% \end{itemize}
% }
% 
\section{Reducibility}
\frame{
Section 5.1 informal reductions. The book shows many problems are undecidable using the same method. If being able to decide some set $A$ lets you decide $A_{TM}$. Then $A$ must not be decidable, because $A_{TM}$ is not.
~~\\
Some undecidable problems: $A_{TM}, HALT_{TM}, E_{TM}, REGULAR_{TM}, EQ_{TM}, E_{LBA}, ALL_{CFG}$.
}

\frame{
Section 5.3 introduces a mathematical formalism for this concept.
~\\~\\
Mapping reducibility: Given two languages $A$ and $B$, $A$ is mapping reducible to $B$ ($A \le_M B$) if there is a \emph{computable} function $f$ such that:
$$w \in A \iff f(w) \in B$$
Computable function $f$ means some Turing machine can take in $w$ and leave $f(w)$ on the tape for all $w$.
}

\frame{
$$A = \{w \in \Sigma^* ~|~ w \text{ contains an even number of 0s}\}$$
$$B = \{w \in 0^* ~|~ w \text{ contains an even number of 0s}\}$$
~\\
Describe a mapping reduction.
}

\frame{
$$A = \{w \in \Sigma^* ~|~ w \text{ contains an even number of 0s}\}$$
$$B = \{w \in 0^* ~|~ w \text{ contains an \emph{odd} number of 0s}\}$$
~\\
Describe a mapping reduction.
}

\frame{
Consider a reduction from $ALL_{DFA}$ to $E_{DFA}$.
}

\frame{
$ALL_{DFA} \le_M E_{DFA}$ and $E_{DFA}$ is Turing decidable, so $ALL_{DFA}$ is Turing decidable. In general $A \le_{M} B$ and $B$ decidable implies $A$ is decidable. What can we conclude if $A \le_M B$:
\begin{itemize}
\item $A$ is decidable $\implies$ ?
\item $B$ is undecidable $\implies$ ?
\item $A$ is undecidable $\implies$ ?
\end{itemize}
}

\frame{
If $A \le_M B$, then 
\begin{itemize}
\item $A$ is decidable $\implies$ nothing about $B$
\item $B$ is decidable $\implies A$ is decidable
\item $A$ is undecidable $\implies B$ is undecidable
\item $B$ is undecidable $\implies$ nothing about $A$
\end{itemize}
}

\frame{
Describe a mapping reduction from $A_{TM}$ to $E_{TM}$.
}

\frame{
Need to describe a computable process, $f$, such that: 
$$(M, w) \in A_{TM} \iff f((M,w)) \in E_{TM}$$
\pause
$E_{TM}$ expects as input a Turing machine, so we must describe how to construct a new Turing machine $M'$ from $(M,w)$ such that:
$$(M, w) \in A_{TM} \iff M' \in E_{TM}$$
}

\frame{
The following Turing machine takes in $(M,w)$ and converts it a suitable $M'$.
~\\
~\\
\begin{tabular}{ll}
$T$=& On input $(M,w)$. \\ 
& 1. Construct the following Turing machine: \\
& \begin{tabular}{ll}
$S$=& On input $x$: \\ 
& 1. If $x \ne w$, \emph{reject}.\\
& 2. Else, run $M$ and $w$ and accept if it does.
\end{tabular}
\end{tabular}
\pause
~\\
If we could decide $E_{TM}$, then we could decide $A_{TM}$ by taking its input and constructing $S$ and then running our decider for $E_{TM}$ on $S$. $A_{TM}$ cannot be decided though, so a decider for $E_{TM}$ cannot exist. $E_{TM}$ is undecidable.
\pause
~\\
or $A_{TM} \le_M E_{TM}$ and $A_{TM}$ is undecidable so $E_{TM}$ is undecidable.
}



% \frame{
% A $2DFA$ that has two read heads instead of one, they start from both ends of the input and work toward the other end.
% }
% 
% 
% \frame{
% $$ \$ C_1 \$ $$ 
% $$ \# C_1 \$ C_2 \$ C_2^R \# $$
% $$ \# C_1 \# C_3  \$ C_3 \$ C_2^R \# C_2^R \# $$
% $$ \# C_1 \# C_3  \# C_3 \$ C_4 \$ C_4^R \# C_2^R \# C_2^R \# $$
% $$ \# C_1 \# C_3  \# C_3 \# C_5 \$ C_5 \$ C_4^R \# C_4^R \# C_2^R \# C_2^R \# $$
% $$\vdots$$
% $$ \# C_1 \# C_3  \# C_3 \# \hdots \# C_n \$ C_n \$ C_{n-1}^R \# C_{n-1} \# \hdots \# C_4^R  \# C_2^R \# C_2^R \# $$
% $$ \# C_1 \# C_3  \# C_3 \# \hdots \# C_{n-1} \$ C_{n-1} \$ C_{n} \# C_{n}^R \# \hdots \# C_4^R \# C_4^R \# C_2^R \# C_2^R \# $$
% $$\vdots$$
% }
\end{document}


%\begin{VCPicture}{(-1,1)(4,1)}
%\State[p]{(0,0)}{A} \State[q]{(3,0)}{B}
%\Initial{A} \Final{B}
%\EdgeL{A}{B}{a}
%\end{VCPicture}

%\begin{VCPicture}{(-1,1)(4,1)}
%\State[p]{(0,0)}{A} \FinalState[q]{(3,0)}{B}
%\Initial{A}
%\EdgeL{A}{B}{a}
%\end{VCPicture}
